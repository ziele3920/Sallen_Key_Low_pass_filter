\documentclass[a4paper,12pt]{article}
\usepackage[utf8]{inputenc}
\usepackage[MeX]{polski}
\usepackage[table,xcdraw]{xcolor}
\usepackage[utf8]{inputenc}
\usepackage[T1]{fontenc}
\usepackage{graphicx}
\usepackage{epstopdf}
\usepackage{color}
\usepackage{mathtools}

%%%%%%%%%%%%%%%%%%%%%%%%%%%%%%%%%%%%%%%%%%%%%%%%%%%%%%%%%%%%%STRONA TYTULOWA%%%%%%%%%%%%%%%%%%%%%%%%%%%%%%%%%%%%%%%%%%%%%%%%%%%%%%%%%%%%%%%%%%%%%%%%
\title{\Huge \textbf{Sallen-Key Low-pass Filter\\[0.3in]} 
  \huge Informatyka \\[0.2in]
  \LARGE Dokumentacja projektu
}
\date{21.01.2017}
\author{
 	\quad	\\
  Arkadiusz Ziółkowski\\
}

\begin{document}
\maketitle
\pagebreak

\tableofcontents
\pagebreak


\section{Założenia projektowe}

	\begin{enumerate}
		\item Obliczanie wartości dwóch elementów pasywnych (rezystancji rezystora i 
		pojemności kondensatora) filtru dolnoprzepustowego o strukturze Sallen-Key'a
		na podstawie podanych przez użytkownika dwóch wybranych wartości elementów 
		pasywnych (jednego rezystora i jednego kondensatora) oraz częstotliwości
		granicznej.
		\item Rysowanie charakterystyki amplitudowo-fazowej dla filtru o wyliczonych
		 parametrach.
		\item Możliwość wczytania wielu konfiguracji filtru z pliku .xls, wyliczenia 
		brakujących wartości (pojemności kondensatora i oporu rezystora) i zapisania
		obliczonych konfiguracji w nowym pliku .xls.
	\end{enumerate}

\section{Ogólny mechanizm działania}

	Mechanizm działania programu opiera się o przepływ danych pomiędzy interfejsem
	użytkownika stanowiącym widok aplikacji a jednostką obliczeniową. 
	Wartości wprowadzane przez użytkownika są wstępnie walidowane względem typu
	wprowadzonych danych przez moduł walidacji danych a następnie zostają przekazane
	do jednostki obliczeniowej.
	Jednostka obliczeniowa implementuje algorytm pozwalający
	wyznaczyć wartości elementów pasywnych filtru po wykonaniu, którego przekazuje
	obliczone wartości do widoku w celu prezentacji ich użytkownikowi w postaci liczbowej
	oraz graficznej (charakterystyki Bodego). \\
	Możliwe jest również skorzystanie z opcji pracowania na plikach .xls gdzie zastępowany 
	jest fragment pobierania danych od użytkownika i prezentacji ich użytkownikowi  poprzez
	moduł wczytujący dane z pliku i zapisujący wyniki do pliku.

\section{Opis funkcji dodatkowych}

	\begin{enumerate}
		\item Klasa "Data Validator" posiadająca funkcje:
		\begin{itemize}
			\item retval = IsANumber(src) - funkcja sprawdza, czy w polu tekstowym 
			reprezentowanym przez uchwyt "src" występuje dodatnia wartość liczbowa.
			W przypadku spełnienia tego warunku zwraca wartość logiczną true, w przeciwnym
			wypadku wartość false;
			\item retval = IsAText(src) - funkcja sprawdza, czy w polu tekstowym 
			reprezentowanym przez uchwyt "src" występuje wprowadzony przez użytkownika
			niepusty ciąg znaków i różny od "insert value". W przypadku spełnienia tych
			warunków zwraca wartość logiczną true, w przeciwnym
			wypadku wartość false;
		\end{itemize}
		
		\item Klasa "MSKLPF" reprezentuje model danych przechowujący wartości 
		charakteryzujące filtr:
			\begin{itemize}
				\item fc - częstotliwość graniczna filtru
				\item r1 - wartość rezystancji opornika R1 [kOhm]
				\item r2 - wartość rezystancji opornika R2 [kOhm]
				\item c1 - wartość pojemności kondensatora C1 [uF]
				\item c2 - wartość pojemności kondensatora C2 [uF]
				\item H - współczynniki transmitancji filtru
			\end{itemize}
		
		\item Klasa "SKLPF" posiadająca funkcje:
		\begin{itemize}
			\item calculatedModel = Calculate(model) - oblicza wartość elementów
			filtru na podstawie częściowo uzupełnionego modelu (argument model) 
			reprezentowanego przez klasę MSKLPF. Zwraca model uzupełniony o
			obliczone wartości.
		\end{itemize}
		
		\item list = ReadFile(fileName) - funkcja wczytuje z pliku fileName dane 
		filtrów i zwraca wektor modeli (MSKLPF) filtrów.
		
		\item function WriteFile(fileName, data ) - funkcja zapisuje do pliku o nazwie
		fileName dane filtrów wyłuskane z modeli znajdujących się w wektorze data.
		
	\end{enumerate}

\section{Najtrudniejsze elementy w realizacji programu}

	Najtrudniejszym elementem w realizacji programu okazała się walidacja danych
	wprowadzanych przez użytkownika, ponieważ
	należało rozpatrzeć wszystkie możliwości niekompatybilności tych danych z 
	api programu i ograniczeniami fizycznymi (np. niemożność wykonania kondensatora 
	o ujemnej pojemności).

\section{Najciekawsze elementy programu}

	Najciekawszymi elementami programu są klasy MSKLPF, SKLPF, VSKLPF gdyż
	są one klasami w których model danych został oddzielony od logiki oraz widoku
	programu. 

\section{Obsługa programu}



\section{Format danych}

	Dane wejściowe programu są liczbami dodatnimi opisującymi filtr:
	\begin{itemize}
				\item fc - częstotliwość graniczna filtru
				\item r1 - wartość rezystancji opornika R1 [kOhm]
				\item r2 - wartość rezystancji opornika R2 [kOhm]
				\item c1 - wartość pojemności kondensatora C1 [uF]
				\item c2 - wartość pojemności kondensatora C2 [uF]
	\end{itemize}
	
	podczas wprowadzania danych należy pamiętać, że należy podać wartości 
	fc, oraz po jednej ze zbiorów {r1, r2} , {c1, c2}. Program posiada zabezpieczenia
	przez niepoprawnym wprowadzeniu danych.  \\
	W przypadku wprowadzania danych z pliku .xls należy utworzyć 5 kolumn, które 
	reprezentują wartości opisujące filtr w kolejności wyżej wymienionej.
	Wartości elementów do obliczenia muszą w pliku wynosić 0.
	Plik z danymi wyjściowymi reprezentowany jest w podobnie w pięciu kolumnach
	jak plik wejściowy z tą różnicą, że występuje wiersz nagłówkowy opisujący kolumny.

\section{Możliwy rozwój projektu}

	\begin{enumerate}
		\item Dodanie opcji generowania charakterystyk Bodego dla wprowadzonych przez 
		użytkownika kompletnych danych opisujących filtr.
		\item Wprowadzenie algorytmu doboru elementów pasywnych filtru spośród wartości 
		należących do istniejących szeregów np. E24.
	\end{enumerate}

\end{document}

