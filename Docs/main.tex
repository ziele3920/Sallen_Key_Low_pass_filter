\documentclass[a4paper,12pt]{article}
\usepackage[utf8]{inputenc}
\usepackage[MeX]{polski}
\usepackage[table,xcdraw]{xcolor}
\usepackage[utf8]{inputenc}
\usepackage[T1]{fontenc}
\usepackage{graphicx}
\usepackage{epstopdf}
\usepackage{color}
\usepackage{mathtools}

%%%%%%%%%%%%%%%%%%%%%%%%%%%%%%%%%%%%%%%%%%%%%%%%%%%%%%%%%%%%%STRONA TYTULOWA%%%%%%%%%%%%%%%%%%%%%%%%%%%%%%%%%%%%%%%%%%%%%%%%%%%%%%%%%%%%%%%%%%%%%%%%
\title{\Huge \textbf{Sallen-Key Low-pass Filter\\[0.3in]} 
  \huge Informatyka \\[0.2in]
  \LARGE Dokumentacja projektu
}
\date{21.01.2017}
\author{
 	\quad	\\
  Arkadiusz Ziółkowski\\
}

\begin{document}
\maketitle
\pagebreak

\tableofcontents
\pagebreak


\section{Założenia projektowe}

	\begin{enumerate}
		\item Obliczanie wartości dwóch elementów pasywnych (rezystancji rezystora i 
		pojemności kondensatora) filtru dolnoprzepustowego o strukturze Sallen-Key'a
		na podstawie podanych przez użytkownika dwóch wybranych wartości elementów 
		pasywnych (jednego rezystora i jednego kondensatora) oraz częstotliwości
		granicznej.
		\item Rysowanie charakterystyki amplitudowo-fazowej dla filtru o wyliczonych
		 parametrach.
		\item Możliwość wczytania wielu konfiguracji filtru z pliku .xls, wyliczenia 
		brakujących wartości (pojemności kondensatora i oporu rezystora) i zapisania
		obliczonych konfiguracji w nowym pliku .xls.
	\end{enumerate}

\section{Ogólny mechanizm działania}

	

\section{Opis funkcji dodatkowych}
\section{Najtrudniejsze elementy w realizacji programu}
\section{Najciekawsze elementy programu}
\section{Obsługa programu}
\section{Format danych}
\section{Możliwy rozwój projektu}


\end{document}

